\section{What are the main outcomes, results, conclusions of the assignment?}

Simon has been a Fairphone costumer for a while, and he as been recently asked to become a Trusted Distributor for the company. He accepted immediately since he enjoys helping people, and having favours with the company did not seem like a bad deal for lending a hand occasionally. 

His day begins as usual. He wakes up and goes to work. Not for Fairphone, his usual paying job. After lunch, he receives a notification. There is a costumer in distress! Her name is Margaret and she is a 68 year old lady and she cracked her screen! Fortunately she is not in a rush, therefore Simon can accept her help request and schedule an appointment at a suitable time and location for both through the Diagnostic App built into all Fairphones. This is how the ``One person, multiple roles'' card applies to the project.

They agree to meet at Margarets' house since she is elderly and will definitely require help replacing the cracked screen. When Simon is done working, he picks the required material for the service. He picks up his Kit provided to him by Fairphone and goes over to her house. She recognises him as a the TD she contacted from his profile picture, but if more identification were to be required he would have a business card on him. Due to Fairphone requirements the repair may not take longer than 30 minutes. When the screen is fixed, Margaret receives a bill which is credited to the company and Simon takes home the old broken screen, which is to be sent back to Fairphone for either repairing or recycling. This last part applies to the ``Changing of hands'' card.  

Summing up, Simon is a stakeholder when he is helping a costumer, whereas when he goes about his own life he is simply a costumer. This forms and advantage when trying to reduce warehouse and store costs. In this scenario no Fairphone store is required for Margaret to fix her phone. Used (this service also applies to hardware updates) or broken parts are returned to the warehouse where they are recycled or passed to someone else, perhaps at a reduced price.
