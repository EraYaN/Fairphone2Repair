\section{How did you arrive at these outcomes, results, conclusions? Give a brief description of the process you went through. List any sources you consulted; f.i. internet}

The main outcome, results and conclusions were affected by the initial judgements and assumptions of the project. First of all, Fairphone is a very young company, thus they don't have as many resources as other Tech Giants. However, they have stated on their wesite that they are going to reach their break-even point soon, so this lead to the assumption that minimum financial use would be optimal for this case. With this efficient spending mentality, it seemed the best option to look into something that would not cost a lot to put into place and would sustain itself. Introducing the TD system is not only going to help improve the repair experience, but also Fairphones financial aspects. This is because the TDs work as volunteers, and they can play different roles in the system. One of these roles is providing a repairing service to the costumers. This takes away the necessity for repair centres and storage rooms. Their second role is to collect outdated or broken hardware and to send it back to Fairphone for re-selling at a lower price or recycling.