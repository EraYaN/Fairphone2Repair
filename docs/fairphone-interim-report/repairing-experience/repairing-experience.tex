%!TEX program = xelatex+makeindex+bibtex
%!TEX spellcheck = en_GB
\documentclass[final]{report} %scrreprt of scrartcl
%!TEX program=xelatex+makeindex+bibtex
% Include all project wide packages here.
\usepackage{polyglossia}
\setmainlanguage{english}
\usepackage[strict,autostyle]{csquotes}
%\usepackage[svgnames]{xcolor}
\usepackage{graphicx}
%\usepackage{epstopdf}
%\usepackage{pdfpages}
\usepackage{caption}
\usepackage[list=true]{subcaption}
\usepackage{float}
\usepackage{standalone}
\usepackage{import}
\usepackage{tocloft}
%\usepackage{wrapfig}
%\usepackage{authblk}
%\usepackage{array}
%\usepackage{multicol}
%\usepackage{multirow}
\usepackage{booktabs}
\usepackage[toc,page,title,titletoc]{appendix}
\usepackage{xunicode}
\usepackage{fontspec}
%\usepackage{tikz}
%\usepackage{pgfplots}
\usepackage[binary-units]{siunitx}
\usepackage[parfill]{parskip}
\usepackage[pages=some]{background}
%\usepackage[absolute]{textpos}
\usepackage{titlepic}
%\pgfplotsset{compat=newest}
%\pgfplotsset{plot coordinates/math parser=false}
%\newlength\figureheight 
%\newlength\figurewidth
\usepackage{mathtools}
\usepackage{unicode-math}
\usepackage{rotating}
\usepackage{fancyhdr}
\usepackage[compact]{titlesec}
\usepackage{enumitem}
\usepackage{verbatim}
%\usepackage{blindtext}
%\usepackage[margin=3.5cm,headheight=35pt]{geometry}
\usepackage[
    backend=bibtexu,
	texencoding=utf8,
bibencoding=utf8,
    style=ieee,
    sortlocale=en_GB,
    language=auto
]{biblatex}
\usepackage{listings}
%\usepackage{wrapfig}
\usepackage{fullpage}
\newcommand{\includecode}[4][c]{\lstinputlisting[caption=#2, escapechar=, style=#1,label=#4]{#3}}
\newcommand{\superscript}[1]{\ensuremath{^{\textrm{#1}}}}
\newcommand{\subscript}[1]{\ensuremath{_{\textrm{#1}}}}


\newcommand{\chapternumber}{\thechapter}
\renewcommand{\appendixname}{Appendix}
\renewcommand{\appendixtocname}{Appendices}
\renewcommand{\appendixpagename}{Appendices}

\usepackage[hidelinks]{hyperref} %<--------ALWAYS LAST

\input{../../library/style.tex}
\addbibresource{../../library/bibliography.bib}
\begin{document}
\chapter{Repairing Experience} %TODO Ronan
\label{ch:repairing-experience}
\section{Current system}
One of the final things the customer is going to encounter is the repairing experience.
This is the process by where after they have received the replacement part and then proceed install and hopefully fix their phone.
The current system offers two options if the phone is in need of repair.
\begin{enumerate}
\item The phone is delivered to the central repair workshop
\item DIY repair the phone, following the iFixit tutorial
\end{enumerate}
Both of these have these positives and negatives, the main ones being that it is not efficient to send back each broken phone and then resend it to the customer, the second the user cannot ask questions easily if they get stuck repairing the phone or are missing additional parts that were not apparent when the original debugging was being done. 
This allows for a new solution to be designed that provides reduced delivery impact but increases the amount of repair feedback to customers from any technology background (both experts and technophobes).
These alterations can be implemented via the use of systems and products but it is important that the fundamentals held by Fairphone are maintained. 

\section{Trusted Distributer solution experience}
The proposed idea is to create a repair experience that shows the customer that the product they have picked is both fair and still as efficient as other leading companies.
As mentioned above the current proposal is the Trusted Distributor system.
This concept idea encourage a user interaction that allows the customer to decide how they repair their phone and have immediate feedback if they decided to opt of the DIY route.
This will both further the knowledge of technophobes and creating a smarter Fairphone community.
The experience will be less stressful than sending the phone of top be repaired as it produces immediate results without having multiple forms to fill out and having to go through the weighting process where the user do not have their phone.
This is both beneficial for the customer, quicker service, and beneficial for the company as they only have to macro-manage and not worry about micromanaging.

\end{document}