%!TEX program = xelatex+makeindex+bibtex
%!TEX spellcheck = en_GB
\documentclass[final]{scrreprt} %scrreprt of scrartcl
%!TEX program=xelatex+makeindex+bibtex
% Include all project wide packages here.
\usepackage{polyglossia}
\setmainlanguage{english}
\usepackage[strict,autostyle]{csquotes}
%\usepackage[svgnames]{xcolor}
\usepackage{graphicx}
%\usepackage{epstopdf}
%\usepackage{pdfpages}
\usepackage{caption}
\usepackage[list=true]{subcaption}
\usepackage{float}
\usepackage{standalone}
\usepackage{import}
\usepackage{tocloft}
%\usepackage{wrapfig}
%\usepackage{authblk}
%\usepackage{array}
%\usepackage{multicol}
%\usepackage{multirow}
\usepackage{booktabs}
\usepackage[toc,page,title,titletoc]{appendix}
\usepackage{xunicode}
\usepackage{fontspec}
%\usepackage{tikz}
%\usepackage{pgfplots}
\usepackage[binary-units]{siunitx}
\usepackage[parfill]{parskip}
\usepackage[pages=some]{background}
%\usepackage[absolute]{textpos}
\usepackage{titlepic}
%\pgfplotsset{compat=newest}
%\pgfplotsset{plot coordinates/math parser=false}
%\newlength\figureheight 
%\newlength\figurewidth
\usepackage{mathtools}
\usepackage{unicode-math}
\usepackage{rotating}
\usepackage{fancyhdr}
\usepackage[compact]{titlesec}
\usepackage{enumitem}
\usepackage{verbatim}
%\usepackage{blindtext}
%\usepackage[margin=3.5cm,headheight=35pt]{geometry}
\usepackage[
    backend=bibtexu,
	texencoding=utf8,
bibencoding=utf8,
    style=ieee,
    sortlocale=en_GB,
    language=auto
]{biblatex}
\usepackage{listings}
%\usepackage{wrapfig}
\usepackage{fullpage}
\newcommand{\includecode}[4][c]{\lstinputlisting[caption=#2, escapechar=, style=#1,label=#4]{#3}}
\newcommand{\superscript}[1]{\ensuremath{^{\textrm{#1}}}}
\newcommand{\subscript}[1]{\ensuremath{_{\textrm{#1}}}}


\newcommand{\chapternumber}{\thechapter}
\renewcommand{\appendixname}{Appendix}
\renewcommand{\appendixtocname}{Appendices}
\renewcommand{\appendixpagename}{Appendices}

\usepackage[hidelinks]{hyperref} %<--------ALWAYS LAST

\input{../../library/style.tex}
\addbibresource{../../library/bibliography.bib}
\begin{document}
\chapter{Trusted Distributors} %TODO Vincent
\label{ch:trusted-distributors}
As a part of the distribution system, the inclusion of the costumers in the process was the most preferred option as a result of the brainstorm in section \ref{ch:initial-research}. This section describes how this is made possible.

Taking Simon from chapter \ref{ch:breaking-experience} into consideration, he would like to contribute to Fairphones' good doing by being the bridge between the costumer and the company. His function will involve two aspects: Providing hardware to costumers and provide the service, if necessary, to install it. First of, he has to be selected to be come a Trusted Distributor (TD).

\section{Becoming a Trusted Distributor}
Considering costumers would figuratively become the bridge between the costumer and the company, they need first of all, to be reliable and loyal. One of the possible ways to get selected, is by being a helpful member on the Fairphone forums, prepared to help people and having a high activity. This would result into being asked by the company to fill that position. An other way could be through application. Just as you would for a normal job. 

Of course not everyone can become a TD. This function should be special, since a TD is provided with a repair kit, which includes one or two complete dissembled phones, with instructions on how to take them apart and put them back together. The amount of TDs per city is defined by the amount of Fairphone owners in that city. The more people own a phone, the more TDs are required to meet the supply/demand requirements. This is determined through the Fairphone mobile registration which takes place when the product is purchased. On the registration, the motivation for the question will be stated, for example: ``\textit{Fairphone requires this information in order to help you through the Trusted Distributor network.}''  

%A percentage of FairPhone owners in a certain city (determined by initial phone registration). Since its mainly Europe, if someone has FairPhone over seas, direct shipping.

%Consumer can DIY, if not, Trusted Dealer can replace the broken hardware for virtual reward (or compensation decided by Dealer and Consumer, but this not up to FairPhone)

\section{Regulations}\label{sec:regulations}
As mentioned before, the TD has to provide a service to the costumers in need. This means that he will be given two complete sets of hardware, which are still owned by the company. It would therefore be theft if he decided to keep a part for himself. Furthermore, the TD might try to get away with getting all the advantages of being a TD and not do his job by saying he is always not available when someone needs help (dodging). This section is about finding the loopholes that might make this system corrupt, and instead make it efficient and pleasant for everyone.

\subsection{Responsibility}
To avoid ``dodging'', a self regulating rule could be set up. The rule states that if a TD has not helped anyone in a certain amount of time, his role is taken away with the reasoning that his function is unnecessary. Not having provided any service could be the results of either dodging, or perhaps that his location is unfavoured by the costumers and no one lives near enough to go that specific TD. Once the function has been revoked, the TD's tool-kit has to be sent back to the company, so that it may be given to someone else, or used in a different way.

A second matter of concern are the differentiations between the illegal act of theft, and simply losing the kit. To start off, the kit has to be insured by either the company or the TD. Since the insurance system already has a lot of experience it is best to let them deal with such matters. However, if a component turns out to be lost, a one-time strike system can be put into place. Just like when an employee would be fired, the second time he does something wrong. If it was actually stolen by the TD, the same actions as catching and employee stealing would be undertaken. 

\section{Privileges} 
%Make a virtual reward system:
%•	Points
%•	VIP
%•	First to know about new releases
%•	Discount on FairPhone products (10-15%)
%•	Beta testing
%•	T-shirt
%•	Tool-kit
%•	(Other low cost, happy making things)

%Punish when not following rules or stealing.



\end{document}