%!TEX program = xelatex+makeindex+bibtex
%!TEX spellcheck = en_GB
\documentclass[final]{scrreprt} %scrreprt of scrartcl
%!TEX program=xelatex+makeindex+bibtex
% Include all project wide packages here.
\usepackage{fullpage}
\usepackage{polyglossia}
\setmainlanguage{english}
\usepackage{csquotes}
\usepackage{graphicx}
\usepackage{epstopdf}
\usepackage{pdfpages}
\usepackage{caption}
\usepackage[list=true]{subcaption}
\usepackage{float}
\usepackage{standalone}
\usepackage{import}
\usepackage{tocloft}
\usepackage{wrapfig}
\usepackage{authblk}
\usepackage{array}
\usepackage{booktabs}
\usepackage[toc,page,title,titletoc]{appendix}
\usepackage{xunicode}
\usepackage{fontspec}
\usepackage{pgfplots}
\usepackage{SIunitx}
\usepackage{units}
\pgfplotsset{compat=newest}
\pgfplotsset{plot coordinates/math parser=false}
\newlength\figureheight 
\newlength\figurewidth
\usepackage{amsmath}
\usepackage{mathtools}
\usepackage{unicode-math}
\usepackage{rotating}
\usepackage{fancyhdr}
\usepackage{titlesec}
\usepackage{blindtext}
\usepackage{color}
\usepackage[margin=3.5cm,headheight=35pt]{geometry}
\usepackage[
    backend=bibtexu,
	texencoding=utf8,
bibencoding=utf8,
    style=ieee,
    sortlocale=en_US,
    language=auto
]{biblatex}
\usepackage{listings}
\usepackage{wrapfig}
\newcommand{\includecode}[4][c]{\lstinputlisting[caption=#2, escapechar=, style=#1,label=#4]{#3}}
\newcommand{\superscript}[1]{\ensuremath{^{\textrm{#1}}}}
\newcommand{\subscript}[1]{\ensuremath{_{\textrm{#1}}}}


\newcommand{\chapternumber}{\thechapter}
\renewcommand{\appendixname}{Appendix}
\renewcommand{\appendixtocname}{Appendices}
\renewcommand{\appendixpagename}{Appendices}

\usepackage[hidelinks]{hyperref} %<--------ALTIJD ALS LAATSTE

%!TEX program=xelatex+makeindex+bibtex
\renewcommand{\familydefault}{\sfdefault}

\setmainfont[Ligatures=TeX]{Calibri}
\setmathfont{Asana Math}
\setmonofont{Lucida Console}

%\definecolor{chapterbarcolor}{cmyk}{.52,.32,0,0}
%\definecolor{footrulecolor}{cmyk}{.52,.32,0,0}

\definecolor{chapterbarcolor}{gray}{0.75}
\definecolor{footrulecolor}{gray}{0.75}

\fancypagestyle{plain}{%
  \fancyhf{}    
  \fancyfoot[L]{\ifnum\value{chapter}>0 \chaptername\ \thechapter. \fi}
  \fancyfoot[C]{\thepage}
  \fancyfoot[R]{\small \today}
  \renewcommand{\headrulewidth}{0pt}
  \renewcommand{\footrulewidth}{2pt}
  \renewcommand{\footrule}{\hbox to\headwidth{%
  \color{footrulecolor}\leaders\hrule height \footrulewidth\hfill}}
}

\pagestyle{plain}

\newcommand{\hsp}{\hspace{20pt}}
\titleformat{\chapter}[hang]{\Huge\bfseries}{\chapternumber\hsp\textcolor{chapterbarcolor}{|}\hsp}{0pt}{\Huge\bfseries}
\titlespacing{\chapter}{0pt}{0pt}{1pt}
\renewcommand{\familydefault}{\sfdefault}
\renewcommand{\arraystretch}{1.2}
\setlength{\headheight}{0pt} 
\setlength\parindent{0pt}
\setlength{\parskip}{0.3cm plus4mm minus3mm}
\setlength\cftaftertoctitleskip{5pt}
\setlength\cftbeforetoctitleskip{20pt}

%For code listings
\definecolor{black}{rgb}{0,0,0}
\definecolor{browntags}{rgb}{0.65,0.1,0.1}
\definecolor{bluestrings}{rgb}{0,0,1}
\definecolor{graycomments}{rgb}{0.4,0.4,0.4}
\definecolor{redkeywords}{rgb}{1,0,0}
\definecolor{bluekeywords}{rgb}{0.13,0.13,0.8}
\definecolor{greencomments}{rgb}{0,0.5,0}
\definecolor{redstrings}{rgb}{0.9,0,0}
\definecolor{purpleidentifiers}{rgb}{0.01,0,0.01}


\lstdefinestyle{csharp}{
language=[Sharp]C,
showspaces=false,
showtabs=false,
breaklines=true,
showstringspaces=false,
breakatwhitespace=true,
escapeinside={(*@}{@*)},
columns=fullflexible,
commentstyle=\color{greencomments},
keywordstyle=\color{bluekeywords}\bfseries,
stringstyle=\color{redstrings},
identifierstyle=\color{purpleidentifiers},
basicstyle=\ttfamily\small}

\lstdefinestyle{c}{
language=C,
showspaces=false,
showtabs=false,
breaklines=true,
showstringspaces=false,
breakatwhitespace=true,
escapeinside={(*@}{@*)},
columns=fullflexible,
commentstyle=\color{greencomments},
keywordstyle=\color{bluekeywords}\bfseries,
stringstyle=\color{redstrings},
identifierstyle=\color{purpleidentifiers},
}

\lstdefinestyle{matlab}{
language=Matlab,
showspaces=false,
showtabs=false,
breaklines=true,
showstringspaces=false,
breakatwhitespace=true,
escapeinside={(*@}{@*)},
columns=fullflexible,
commentstyle=\color{greencomments},
keywordstyle=\color{bluekeywords}\bfseries,
stringstyle=\color{redstrings},
identifierstyle=\color{purpleidentifiers}
}

\lstdefinestyle{vhdl}{
language=VHDL,
showspaces=false,
showtabs=false,
breaklines=true,
showstringspaces=false,
breakatwhitespace=true,
escapeinside={(*@}{@*)},
columns=fullflexible,
commentstyle=\color{greencomments},
keywordstyle=\color{bluekeywords}\bfseries,
stringstyle=\color{redstrings},
identifierstyle=\color{purpleidentifiers}
}

\lstdefinestyle{xaml}{
language=XML,
showspaces=false,
showtabs=false,
breaklines=true,
showstringspaces=false,
breakatwhitespace=true,
escapeinside={(*@}{@*)},
columns=fullflexible,
commentstyle=\color{greencomments},
keywordstyle=\color{redkeywords},
stringstyle=\color{bluestrings},
tagstyle=\color{browntags},
morestring=[b]",
  morecomment=[s]{<?}{?>},
  morekeywords={xmlns,version,typex:AsyncRecords,x:Arguments,x:Boolean,x:Byte,x:Char,x:Class,x:ClassAttributes,x:ClassModifier,x:Code,x:ConnectionId,x:Decimal,x:Double,x:FactoryMethod,x:FieldModifier,x:Int16,x:Int32,x:Int64,x:Key,x:Members,x:Name,x:Object,x:Property,x:Shared,x:Single,x:String,x:Subclass,x:SynchronousMode,x:TimeSpan,x:TypeArguments,x:Uid,x:Uri,x:XData,Grid.Column,Grid.ColumnSpan,Click,ClipToBounds,Content,DropDownOpened,FontSize,Foreground,Header,Height,HorizontalAlignment,HorizontalContentAlignment,IsCancel,IsDefault,IsEnabled,IsSelected,Margin,MinHeight,MinWidth,Padding,SnapsToDevicePixels,Target,TextWrapping,Title,VerticalAlignment,VerticalContentAlignment,Width,WindowStartupLocation,Binding,Mode,OneWay,xmlns:x}
}

%defaults
\lstset{
basicstyle=\ttfamily\scriptsize ,
extendedchars=false,
numbers=left,
numberstyle=\ttfamily\tiny,
stepnumber=1,
tabsize=4,
numbersep=5pt
}
\addbibresource{../../library/bibliography.bib}
\begin{document}
\chapter{Initial Research}
\label{ch:initial-research}
%TODO Lian & Wouter
Fairphone is a social enterprise that started with the goal of opening up the supply chain, understanding how products are make and creating a better connection between people and the things they own. Fair phone raises consumer's awareness about the social and environmental impacts of the electronics. Fair phone as a company, focuses more on the ethical aspects rather than make technological improvements and break through. Fair phone is not about the phone, it's about the movement. 

We as a group, were given the assignment to redefine the repair experience of the fair phone more engaging and enjoyable. However, the whole project started with the whole group going in the wrong direction. At the beginning of the project, we thought we were going to redesign the phone in order to make it more repairable, we thought it was about how easy the phone can be repaired but that was not the task that was given to us. Hence we started to come up with different ideas how to make the phone easier to repair, we started to redesign the phone. Our main direction was modularity, making the phone modular could be the ultimate solution to repairability. However, it is not feasible with the current technology yet. Thus we started to think of partially modular. Making the parts that breaks more easily modular can ease the process of repair by miles. Through some research, we found that the majority of the consumers have 3 main parts in which they break, the headphone jack, the USB and the screen. We came up with a couple of solution that could make partially modular feasible and that's when we found out that we went to the wrong direction. 

Starting off with these intentions we focused on making the hardware of the device easier to repair. We did research on the device and made a list of negative points and a list of suggestions. 
The negative points were:										
\begin{itemize}
	\item The product is relatively expensive when compare the hardware at similar price range
	\item The back case consists of several parts
	\item The headphone jack is connected to the motherboard
	\item The PCB is inefficient 	
\end{itemize} 

Our suggestions were:
\begin{itemize}
	\item Some of the parts on the motherboard can be modularised
	\item The Micro USB port should be more resilient or modularised
	\item The amount of used materials (plastic) could be reduced by making the design a bit slimmer
\end{itemize}



Instead, our task was to recreate a better service in which you go through after your phone is broken. 

After the first meeting with Miquel, the group had a better understanding of the assigned task and started to brainstorm about different possibilities to solve this problem. 

The whole system of getting the phone repaired cannot be changed entirely, however, each individual components within the system can be approached in such a way to improve the repair experience. The system consists of 4 main components, detect broken parts in the phone, ordering parts for repair, pick up parts for repair and fixing the phone.  

How to find out your phone is broken?
\begin{itemize}
	\item Factory application
	\item Computer program
	\item Hardware diagnosis (connector on the motherboard or USB)
	\item Hardware diagnosis on battery
	\item Constant hardware monitoring with notification
	\item Hardware docking station
	\item Hardware diagnosis
	\item Factory quality control machine 
	\item Always online Fairphone connection
\end{itemize}

How to order the parts?
\begin{itemize}
	\item Through the official website
	\item Factory application
	\item Parts stocked at Trusted Distributor
	\item Order through text
	\item Automatic part order
	\item Automatic order docking station
\end{itemize}

How to receive parts for repair?
\begin{itemize}
	\item Trusted Distributors stock parts to pick up
	\item Delivery service
	\item Normal mail
	\item Drone delivery
\end{itemize}

How to fix your phone?
\begin{itemize}
	\item iFixit videos
	\item Step by step manuals with pictures
	\item Trusted Distributor repair service
	\item Docking station
	\item Family members with some technological knowledge
\end{itemize}

After brainstorming, several possible solutions were developed, considering financial viability, serviceability and eco-friendliness.

The solutions that suited the criteria best were: 

Financially:
\begin{itemize}
	\item Built in application
	\item Internet or through the application
	\item Self pick up
	\item PDF on the Internet
\end{itemize}

Service:
\begin{itemize}
	\item Self diagnosis
	\item Automatic order
	\item Home delivery
	\item Custom repair
\end{itemize}

Eco-friendly
\begin{itemize}
	\item Built in application
	\item Internet store
	\item Deliver to trusted distributor (TD) and pick up
	\item Trusted distributor fix
\end{itemize}
\end{document}