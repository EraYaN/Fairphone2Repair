%!TEX program = xelatex+makeindex+bibtex
%!TEX spellcheck = en_GB
\documentclass[final]{scrreprt} %scrreprt of scrartcl
%!TEX program=xelatex+makeindex+bibtex
% Include all project wide packages here.
\usepackage{polyglossia}
\setmainlanguage{english}
\usepackage[strict,autostyle]{csquotes}
%\usepackage[svgnames]{xcolor}
\usepackage{graphicx}
%\usepackage{epstopdf}
%\usepackage{pdfpages}
\usepackage{caption}
\usepackage[list=true]{subcaption}
\usepackage{float}
\usepackage{standalone}
\usepackage{import}
\usepackage{tocloft}
%\usepackage{wrapfig}
%\usepackage{authblk}
%\usepackage{array}
%\usepackage{multicol}
%\usepackage{multirow}
\usepackage{booktabs}
\usepackage[toc,page,title,titletoc]{appendix}
\usepackage{xunicode}
\usepackage{fontspec}
%\usepackage{tikz}
%\usepackage{pgfplots}
\usepackage[binary-units]{siunitx}
\usepackage[parfill]{parskip}
\usepackage[pages=some]{background}
%\usepackage[absolute]{textpos}
\usepackage{titlepic}
%\pgfplotsset{compat=newest}
%\pgfplotsset{plot coordinates/math parser=false}
%\newlength\figureheight 
%\newlength\figurewidth
\usepackage{mathtools}
\usepackage{unicode-math}
\usepackage{rotating}
\usepackage{fancyhdr}
\usepackage[compact]{titlesec}
\usepackage{enumitem}
\usepackage{verbatim}
%\usepackage{blindtext}
%\usepackage[margin=3.5cm,headheight=35pt]{geometry}
\usepackage[
    backend=bibtexu,
	texencoding=utf8,
bibencoding=utf8,
    style=ieee,
    sortlocale=en_GB,
    language=auto
]{biblatex}
\usepackage{listings}
%\usepackage{wrapfig}
\usepackage{fullpage}
\newcommand{\includecode}[4][c]{\lstinputlisting[caption=#2, escapechar=, style=#1,label=#4]{#3}}
\newcommand{\superscript}[1]{\ensuremath{^{\textrm{#1}}}}
\newcommand{\subscript}[1]{\ensuremath{_{\textrm{#1}}}}


\newcommand{\chapternumber}{\thechapter}
\renewcommand{\appendixname}{Appendix}
\renewcommand{\appendixtocname}{Appendices}
\renewcommand{\appendixpagename}{Appendices}

\usepackage[hidelinks]{hyperref} %<--------ALWAYS LAST

\input{../../library/style.tex}
\addbibresource{../../library/bibliography.bib}
\begin{document}
\chapter{Breaking Experience}
\label{ch:breaking-experience}
The starting point of the project is a customer with a broken product, and obviously, they all want the same thing: a fast, economic and possibly fun way to repair the phone. This process should possible for everyone, therefore three scenario people have been made up: Peter, Margaret and Simon.

Peter is a 27 year old university graduate, he is good with technology and likes his things updated and at the latest version. Moreover, he likes to this while thinking about the environment, likes the idea fair trade and the respect for human rights since it was a big sector in his study. He chose the product for all those reasons. However, he the phones USB charging port broke, he is almost out of battery and his frustration is growing since he has a date to go to. 

Margaret is 68 year old lady, she goes to church every Sunday and is really involved with human rights activism and the fight against child labour. She has been gifted the phone by her 13 year old grandson, which, despite his young age, as his way with technology and manages to help her out with tech-problems in the weekends. Margaret unfortunately dropped her phone while at a gathering and cracked her screen. The phone still works fine, but since she was already having trouble reading all the small letters on the screen, those cracks aren't making it easier for her. Thankfully her grandson is coming to visit in a few days, and she has told him about the problem.

Simon is a 43 year old man, and works in tech support. He is an active activism-forum user. He is very concerned with the consequences of child labour and the neglect of human rights in some countries. He has a recently purchased a Fairphone and would like to contribute to Fairphones' efforts in making a better world. He already uses his forums to provide tech support in his free time, and would really like to be able to do more than that.

Creating Peter, Margaret and Simon helps creating a process that suits three extreme ends of possible costumers. The challenge is in fact doing so. The best way to start in this case, is by ``brainstorming'' and performing some basic research about the current ways of Fairphone.
\end{document}