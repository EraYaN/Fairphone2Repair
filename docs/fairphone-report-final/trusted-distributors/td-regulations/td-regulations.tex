% !TeX spellcheck = en_GB
% !TEX program = xelatex+makeindex+bibtex
% !TeX TXS-program:pdflatex = txs:///xelatex %
\documentclass[final,a4paper]{report} %scrreprt of scrartcl
%!TEX program=xelatex+makeindex+bibtex
% Include all project wide packages here.
\usepackage{fullpage}
\usepackage{polyglossia}
\setmainlanguage{english}
\usepackage{csquotes}
\usepackage{graphicx}
\usepackage{epstopdf}
\usepackage{pdfpages}
\usepackage{caption}
\usepackage[list=true]{subcaption}
\usepackage{float}
\usepackage{standalone}
\usepackage{import}
\usepackage{tocloft}
\usepackage{wrapfig}
\usepackage{authblk}
\usepackage{array}
\usepackage{booktabs}
\usepackage[toc,page,title,titletoc]{appendix}
\usepackage{xunicode}
\usepackage{fontspec}
\usepackage{pgfplots}
\usepackage{SIunitx}
\usepackage{units}
\pgfplotsset{compat=newest}
\pgfplotsset{plot coordinates/math parser=false}
\newlength\figureheight 
\newlength\figurewidth
\usepackage{amsmath}
\usepackage{mathtools}
\usepackage{unicode-math}
\usepackage{rotating}
\usepackage{fancyhdr}
\usepackage{titlesec}
\usepackage{blindtext}
\usepackage{color}
\usepackage[margin=3.5cm,headheight=35pt]{geometry}
\usepackage[
    backend=bibtexu,
	texencoding=utf8,
bibencoding=utf8,
    style=ieee,
    sortlocale=en_US,
    language=auto
]{biblatex}
\usepackage{listings}
\usepackage{wrapfig}
\newcommand{\includecode}[4][c]{\lstinputlisting[caption=#2, escapechar=, style=#1,label=#4]{#3}}
\newcommand{\superscript}[1]{\ensuremath{^{\textrm{#1}}}}
\newcommand{\subscript}[1]{\ensuremath{_{\textrm{#1}}}}


\newcommand{\chapternumber}{\thechapter}
\renewcommand{\appendixname}{Appendix}
\renewcommand{\appendixtocname}{Appendices}
\renewcommand{\appendixpagename}{Appendices}

\usepackage[hidelinks]{hyperref} %<--------ALTIJD ALS LAATSTE

%!TEX program=xelatex+makeindex+bibtex
\renewcommand{\familydefault}{\sfdefault}

\setmainfont[Ligatures=TeX]{Calibri}
\setmathfont{Asana Math}
\setmonofont{Lucida Console}

%\definecolor{chapterbarcolor}{cmyk}{.52,.32,0,0}
%\definecolor{footrulecolor}{cmyk}{.52,.32,0,0}

\definecolor{chapterbarcolor}{gray}{0.75}
\definecolor{footrulecolor}{gray}{0.75}

\fancypagestyle{plain}{%
  \fancyhf{}    
  \fancyfoot[L]{\ifnum\value{chapter}>0 \chaptername\ \thechapter. \fi}
  \fancyfoot[C]{\thepage}
  \fancyfoot[R]{\small \today}
  \renewcommand{\headrulewidth}{0pt}
  \renewcommand{\footrulewidth}{2pt}
  \renewcommand{\footrule}{\hbox to\headwidth{%
  \color{footrulecolor}\leaders\hrule height \footrulewidth\hfill}}
}

\pagestyle{plain}

\newcommand{\hsp}{\hspace{20pt}}
\titleformat{\chapter}[hang]{\Huge\bfseries}{\chapternumber\hsp\textcolor{chapterbarcolor}{|}\hsp}{0pt}{\Huge\bfseries}
\titlespacing{\chapter}{0pt}{0pt}{1pt}
\renewcommand{\familydefault}{\sfdefault}
\renewcommand{\arraystretch}{1.2}
\setlength{\headheight}{0pt} 
\setlength\parindent{0pt}
\setlength{\parskip}{0.3cm plus4mm minus3mm}
\setlength\cftaftertoctitleskip{5pt}
\setlength\cftbeforetoctitleskip{20pt}

%For code listings
\definecolor{black}{rgb}{0,0,0}
\definecolor{browntags}{rgb}{0.65,0.1,0.1}
\definecolor{bluestrings}{rgb}{0,0,1}
\definecolor{graycomments}{rgb}{0.4,0.4,0.4}
\definecolor{redkeywords}{rgb}{1,0,0}
\definecolor{bluekeywords}{rgb}{0.13,0.13,0.8}
\definecolor{greencomments}{rgb}{0,0.5,0}
\definecolor{redstrings}{rgb}{0.9,0,0}
\definecolor{purpleidentifiers}{rgb}{0.01,0,0.01}


\lstdefinestyle{csharp}{
language=[Sharp]C,
showspaces=false,
showtabs=false,
breaklines=true,
showstringspaces=false,
breakatwhitespace=true,
escapeinside={(*@}{@*)},
columns=fullflexible,
commentstyle=\color{greencomments},
keywordstyle=\color{bluekeywords}\bfseries,
stringstyle=\color{redstrings},
identifierstyle=\color{purpleidentifiers},
basicstyle=\ttfamily\small}

\lstdefinestyle{c}{
language=C,
showspaces=false,
showtabs=false,
breaklines=true,
showstringspaces=false,
breakatwhitespace=true,
escapeinside={(*@}{@*)},
columns=fullflexible,
commentstyle=\color{greencomments},
keywordstyle=\color{bluekeywords}\bfseries,
stringstyle=\color{redstrings},
identifierstyle=\color{purpleidentifiers},
}

\lstdefinestyle{matlab}{
language=Matlab,
showspaces=false,
showtabs=false,
breaklines=true,
showstringspaces=false,
breakatwhitespace=true,
escapeinside={(*@}{@*)},
columns=fullflexible,
commentstyle=\color{greencomments},
keywordstyle=\color{bluekeywords}\bfseries,
stringstyle=\color{redstrings},
identifierstyle=\color{purpleidentifiers}
}

\lstdefinestyle{vhdl}{
language=VHDL,
showspaces=false,
showtabs=false,
breaklines=true,
showstringspaces=false,
breakatwhitespace=true,
escapeinside={(*@}{@*)},
columns=fullflexible,
commentstyle=\color{greencomments},
keywordstyle=\color{bluekeywords}\bfseries,
stringstyle=\color{redstrings},
identifierstyle=\color{purpleidentifiers}
}

\lstdefinestyle{xaml}{
language=XML,
showspaces=false,
showtabs=false,
breaklines=true,
showstringspaces=false,
breakatwhitespace=true,
escapeinside={(*@}{@*)},
columns=fullflexible,
commentstyle=\color{greencomments},
keywordstyle=\color{redkeywords},
stringstyle=\color{bluestrings},
tagstyle=\color{browntags},
morestring=[b]",
  morecomment=[s]{<?}{?>},
  morekeywords={xmlns,version,typex:AsyncRecords,x:Arguments,x:Boolean,x:Byte,x:Char,x:Class,x:ClassAttributes,x:ClassModifier,x:Code,x:ConnectionId,x:Decimal,x:Double,x:FactoryMethod,x:FieldModifier,x:Int16,x:Int32,x:Int64,x:Key,x:Members,x:Name,x:Object,x:Property,x:Shared,x:Single,x:String,x:Subclass,x:SynchronousMode,x:TimeSpan,x:TypeArguments,x:Uid,x:Uri,x:XData,Grid.Column,Grid.ColumnSpan,Click,ClipToBounds,Content,DropDownOpened,FontSize,Foreground,Header,Height,HorizontalAlignment,HorizontalContentAlignment,IsCancel,IsDefault,IsEnabled,IsSelected,Margin,MinHeight,MinWidth,Padding,SnapsToDevicePixels,Target,TextWrapping,Title,VerticalAlignment,VerticalContentAlignment,Width,WindowStartupLocation,Binding,Mode,OneWay,xmlns:x}
}

%defaults
\lstset{
basicstyle=\ttfamily\scriptsize ,
extendedchars=false,
numbers=left,
numberstyle=\ttfamily\tiny,
stepnumber=1,
tabsize=4,
numbersep=5pt
}
\addbibresource{../../../library/bibliography.bib}
\author{Fairphone Repair Group}
\title{Fairphone Repairability}
\date{\today}
\begin{document}
	\section{TD Regulations}
	\label{sec:td-regulations}
	As mentioned before, the TD has to provide a service to the customers in need. This means that Simon will be given one complete sets of the most up-to-date hardware which will however still be owned by the company. It would therefore be considered theft if he decided to keep a part for himself or sell it out of protocol. Furthermore, the TD might try to get away with getting all the advantages (section \ref{sec:td-regulations-advantages}) of being a TD whilst not doing his job by saying he is always unavailable when someone needs help (this concept will be referred to as \textit{dodging}). This section is about finding the loopholes that might make this system corrupt, and instead make it efficient and resourceful for everyone.
	
	\subsection{Responsibilities}
	To avoid ``dodging'', a regulating rule could be set up. The rule states that if a TD has not helped anyone in a certain amount of time, his role is taken away with the reasoning that his function is unnecessary. Not having provided any service could be the results of either dodging, or perhaps that his location is unfavoured by the customers and no one lives near enough to go that specific TD. Once the function has been revoked, the TD's kit has to be sent back to the company, so that it may be given to someone else, or used in a different manner.
	
	A second matter of concern are the differentiations between the illegal act of theft, and simply losing the kit or a component. For starters, the kit has to be insured by either the company or the TD. Since the insurance system already has a lot of experience it is best to let them deal with such matters. However, if a component turns out to be lost, a one-time strike system can be put into place. Just like when an employee, he would be fired the second time he does something wrong or loses something important. If the TD commits theft, the same actions as catching and employee stealing would be undertaken. 
	
	A third way the title of TD may be revoked, is through receiving bad reviews from the helped customers. Basically, if the TD turns out to be unfriendly and rude in the long run, he can be ``fired''.
	
	\subsection{Advantages}
	\label{sec:td-regulations-advantages}
	Being a TD up to this point simply sounds like a lot of work, however a rewards has been considered. These range from virtual to concrete items and even discounts. Below, possible rewards are listed to motivate the unpaid TDs to keep doing their voluntary work:
	
	\begin{itemize}
		\item FairPoints: points that can be accumulated to eventually buy items or Fairphone components (chapter \ref{ch:fairpoints})
		\item Fairphone VIP Forum access
		\item First to know about new releases
		\item Employee Discount on Fairphone products (10-15\%)
		\item Beta testing of soft/hardware
		\item Company/Employee T-shirt
		\item Other gadgets
	\end{itemize}
\end{document}
