% !TeX spellcheck = en_GB
% !TEX program = xelatex+makeindex+bibtex
% !TeX TXS-program:pdflatex = txs:///xelatex %
\documentclass[final,a4paper]{report} %scrreprt of scrartcl
%!TEX program=xelatex+makeindex+bibtex
% Include all project wide packages here.
\usepackage{polyglossia}
\setmainlanguage{english}
\usepackage[strict,autostyle]{csquotes}
%\usepackage[svgnames]{xcolor}
\usepackage{graphicx}
%\usepackage{epstopdf}
%\usepackage{pdfpages}
\usepackage{caption}
\usepackage[list=true]{subcaption}
\usepackage{float}
\usepackage{standalone}
\usepackage{import}
\usepackage{tocloft}
%\usepackage{wrapfig}
%\usepackage{authblk}
%\usepackage{array}
%\usepackage{multicol}
%\usepackage{multirow}
\usepackage{booktabs}
\usepackage[toc,page,title,titletoc]{appendix}
\usepackage{xunicode}
\usepackage{fontspec}
%\usepackage{tikz}
%\usepackage{pgfplots}
\usepackage[binary-units]{siunitx}
\usepackage[parfill]{parskip}
\usepackage[pages=some]{background}
%\usepackage[absolute]{textpos}
\usepackage{titlepic}
%\pgfplotsset{compat=newest}
%\pgfplotsset{plot coordinates/math parser=false}
%\newlength\figureheight 
%\newlength\figurewidth
\usepackage{mathtools}
\usepackage{unicode-math}
\usepackage{rotating}
\usepackage{fancyhdr}
\usepackage[compact]{titlesec}
\usepackage{enumitem}
\usepackage{verbatim}
%\usepackage{blindtext}
%\usepackage[margin=3.5cm,headheight=35pt]{geometry}
\usepackage[
    backend=bibtexu,
	texencoding=utf8,
bibencoding=utf8,
    style=ieee,
    sortlocale=en_GB,
    language=auto
]{biblatex}
\usepackage{listings}
%\usepackage{wrapfig}
\usepackage{fullpage}
\newcommand{\includecode}[4][c]{\lstinputlisting[caption=#2, escapechar=, style=#1,label=#4]{#3}}
\newcommand{\superscript}[1]{\ensuremath{^{\textrm{#1}}}}
\newcommand{\subscript}[1]{\ensuremath{_{\textrm{#1}}}}


\newcommand{\chapternumber}{\thechapter}
\renewcommand{\appendixname}{Appendix}
\renewcommand{\appendixtocname}{Appendices}
\renewcommand{\appendixpagename}{Appendices}

\usepackage[hidelinks]{hyperref} %<--------ALWAYS LAST

\input{../../../library/style.tex}
\addbibresource{../../../library/bibliography.bib}
\author{Fairphone Repair Group}
\title{Fairphone Repairability}
\date{\today}
\begin{document}
	\section{Meeting}
	\label{sec:storyboard-meeting}
	When the part has been ordered and the method of delivery through a TD has been completed (section \ref{sec:td-repairing}), it is up to the customer and the stocked TD to arrange a meeting. This is made very simple through the diagnosing application, where there is no need to exchange contact information, but a chat will start automatically between the stocked TD and the customer. If chatting is not possible as in Margarets case, calling and alternative contact through e-mail or the Fairphone forum is possible too (section \ref{sec:td-repairing}).
	
	It is at this point important to mention an other idea that was not carried forward. The FairPoints system which is better explained in chapter \ref{ch:fairpoints}.
	
	\subsection{Customer to TD}
	In the case that the agreement is to meet at the TD's house, the address will be provided by the application since it will be checked by the company and verify both identities. The only thing that needs to be done is scheduling an appointment. Once actually there, the reparations should not exceed the 30 minutes. When it is all done, the customer can rate and review\footnote{This information will be used by Fairphone in order to advise a Trusted Distributor to a customer and to take action if a TD is not fulfilling his role} the TD with FairPoints, and the more points are awarded (did he offer a drink, was he hospitable etc...) the more credit he receives.
	
	\subsection{TD to customer}
	This works in a similar fashion when the customer asks the TD to come to his house to do the fixing. Of course this would result in additional bonus points in the ratings and the FairPoints accredited to the TD for the extra effort. It is all about providing the best service, so initiative should be rewarded. Just like in section \ref{sec:storyboard-meeting}, the procedure should not take longer than 30 minutes and at the end of the TD gets a review from the customer.
	
	\subsection{DIY}
	In Peters case, he is capable of performing his own repairs so he only needs the new component. He can either pick it up, have the TD deliver it, or have the TD send it to him. If he runs in any complications he can always arrange a Skype session with the TD or consult the manuals on iFixit.
	
	\subsection{Meeting up not possible}
	In the case where meeting up is not possible due to the distance, the component can be sent to the customer (from a warehouse or a TD, which ever comes first) and the customer can consult manuals and iFixit to do the repairs. This will only occur for extraordinary customer abroad, since there will usually be at least one TD in every city with Fairphone users.
\end{document}