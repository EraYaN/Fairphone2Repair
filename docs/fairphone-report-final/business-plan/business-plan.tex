% !TeX spellcheck = en_GB
% !TEX program = xelatex+makeindex+bibtex
% !TeX TXS-program:pdflatex = txs:///xelatex %
\documentclass[final,a4paper]{report} %scrreprt of scrartcl
%!TEX program=xelatex+makeindex+bibtex
% Include all project wide packages here.
\usepackage{fullpage}
\usepackage{polyglossia}
\setmainlanguage{english}
\usepackage{csquotes}
\usepackage{graphicx}
\usepackage{epstopdf}
\usepackage{pdfpages}
\usepackage{caption}
\usepackage[list=true]{subcaption}
\usepackage{float}
\usepackage{standalone}
\usepackage{import}
\usepackage{tocloft}
\usepackage{wrapfig}
\usepackage{authblk}
\usepackage{array}
\usepackage{booktabs}
\usepackage[toc,page,title,titletoc]{appendix}
\usepackage{xunicode}
\usepackage{fontspec}
\usepackage{pgfplots}
\usepackage{SIunitx}
\usepackage{units}
\pgfplotsset{compat=newest}
\pgfplotsset{plot coordinates/math parser=false}
\newlength\figureheight 
\newlength\figurewidth
\usepackage{amsmath}
\usepackage{mathtools}
\usepackage{unicode-math}
\usepackage{rotating}
\usepackage{fancyhdr}
\usepackage{titlesec}
\usepackage{blindtext}
\usepackage{color}
\usepackage[margin=3.5cm,headheight=35pt]{geometry}
\usepackage[
    backend=bibtexu,
	texencoding=utf8,
bibencoding=utf8,
    style=ieee,
    sortlocale=en_US,
    language=auto
]{biblatex}
\usepackage{listings}
\usepackage{wrapfig}
\newcommand{\includecode}[4][c]{\lstinputlisting[caption=#2, escapechar=, style=#1,label=#4]{#3}}
\newcommand{\superscript}[1]{\ensuremath{^{\textrm{#1}}}}
\newcommand{\subscript}[1]{\ensuremath{_{\textrm{#1}}}}


\newcommand{\chapternumber}{\thechapter}
\renewcommand{\appendixname}{Appendix}
\renewcommand{\appendixtocname}{Appendices}
\renewcommand{\appendixpagename}{Appendices}

\usepackage[hidelinks]{hyperref} %<--------ALTIJD ALS LAATSTE

%!TEX program=xelatex+makeindex+bibtex
\renewcommand{\familydefault}{\sfdefault}

\setmainfont[Ligatures=TeX]{Calibri}
\setmathfont{Asana Math}
\setmonofont{Lucida Console}

%\definecolor{chapterbarcolor}{cmyk}{.52,.32,0,0}
%\definecolor{footrulecolor}{cmyk}{.52,.32,0,0}

\definecolor{chapterbarcolor}{gray}{0.75}
\definecolor{footrulecolor}{gray}{0.75}

\fancypagestyle{plain}{%
  \fancyhf{}    
  \fancyfoot[L]{\ifnum\value{chapter}>0 \chaptername\ \thechapter. \fi}
  \fancyfoot[C]{\thepage}
  \fancyfoot[R]{\small \today}
  \renewcommand{\headrulewidth}{0pt}
  \renewcommand{\footrulewidth}{2pt}
  \renewcommand{\footrule}{\hbox to\headwidth{%
  \color{footrulecolor}\leaders\hrule height \footrulewidth\hfill}}
}

\pagestyle{plain}

\newcommand{\hsp}{\hspace{20pt}}
\titleformat{\chapter}[hang]{\Huge\bfseries}{\chapternumber\hsp\textcolor{chapterbarcolor}{|}\hsp}{0pt}{\Huge\bfseries}
\titlespacing{\chapter}{0pt}{0pt}{1pt}
\renewcommand{\familydefault}{\sfdefault}
\renewcommand{\arraystretch}{1.2}
\setlength{\headheight}{0pt} 
\setlength\parindent{0pt}
\setlength{\parskip}{0.3cm plus4mm minus3mm}
\setlength\cftaftertoctitleskip{5pt}
\setlength\cftbeforetoctitleskip{20pt}

%For code listings
\definecolor{black}{rgb}{0,0,0}
\definecolor{browntags}{rgb}{0.65,0.1,0.1}
\definecolor{bluestrings}{rgb}{0,0,1}
\definecolor{graycomments}{rgb}{0.4,0.4,0.4}
\definecolor{redkeywords}{rgb}{1,0,0}
\definecolor{bluekeywords}{rgb}{0.13,0.13,0.8}
\definecolor{greencomments}{rgb}{0,0.5,0}
\definecolor{redstrings}{rgb}{0.9,0,0}
\definecolor{purpleidentifiers}{rgb}{0.01,0,0.01}


\lstdefinestyle{csharp}{
language=[Sharp]C,
showspaces=false,
showtabs=false,
breaklines=true,
showstringspaces=false,
breakatwhitespace=true,
escapeinside={(*@}{@*)},
columns=fullflexible,
commentstyle=\color{greencomments},
keywordstyle=\color{bluekeywords}\bfseries,
stringstyle=\color{redstrings},
identifierstyle=\color{purpleidentifiers},
basicstyle=\ttfamily\small}

\lstdefinestyle{c}{
language=C,
showspaces=false,
showtabs=false,
breaklines=true,
showstringspaces=false,
breakatwhitespace=true,
escapeinside={(*@}{@*)},
columns=fullflexible,
commentstyle=\color{greencomments},
keywordstyle=\color{bluekeywords}\bfseries,
stringstyle=\color{redstrings},
identifierstyle=\color{purpleidentifiers},
}

\lstdefinestyle{matlab}{
language=Matlab,
showspaces=false,
showtabs=false,
breaklines=true,
showstringspaces=false,
breakatwhitespace=true,
escapeinside={(*@}{@*)},
columns=fullflexible,
commentstyle=\color{greencomments},
keywordstyle=\color{bluekeywords}\bfseries,
stringstyle=\color{redstrings},
identifierstyle=\color{purpleidentifiers}
}

\lstdefinestyle{vhdl}{
language=VHDL,
showspaces=false,
showtabs=false,
breaklines=true,
showstringspaces=false,
breakatwhitespace=true,
escapeinside={(*@}{@*)},
columns=fullflexible,
commentstyle=\color{greencomments},
keywordstyle=\color{bluekeywords}\bfseries,
stringstyle=\color{redstrings},
identifierstyle=\color{purpleidentifiers}
}

\lstdefinestyle{xaml}{
language=XML,
showspaces=false,
showtabs=false,
breaklines=true,
showstringspaces=false,
breakatwhitespace=true,
escapeinside={(*@}{@*)},
columns=fullflexible,
commentstyle=\color{greencomments},
keywordstyle=\color{redkeywords},
stringstyle=\color{bluestrings},
tagstyle=\color{browntags},
morestring=[b]",
  morecomment=[s]{<?}{?>},
  morekeywords={xmlns,version,typex:AsyncRecords,x:Arguments,x:Boolean,x:Byte,x:Char,x:Class,x:ClassAttributes,x:ClassModifier,x:Code,x:ConnectionId,x:Decimal,x:Double,x:FactoryMethod,x:FieldModifier,x:Int16,x:Int32,x:Int64,x:Key,x:Members,x:Name,x:Object,x:Property,x:Shared,x:Single,x:String,x:Subclass,x:SynchronousMode,x:TimeSpan,x:TypeArguments,x:Uid,x:Uri,x:XData,Grid.Column,Grid.ColumnSpan,Click,ClipToBounds,Content,DropDownOpened,FontSize,Foreground,Header,Height,HorizontalAlignment,HorizontalContentAlignment,IsCancel,IsDefault,IsEnabled,IsSelected,Margin,MinHeight,MinWidth,Padding,SnapsToDevicePixels,Target,TextWrapping,Title,VerticalAlignment,VerticalContentAlignment,Width,WindowStartupLocation,Binding,Mode,OneWay,xmlns:x}
}

%defaults
\lstset{
basicstyle=\ttfamily\scriptsize ,
extendedchars=false,
numbers=left,
numberstyle=\ttfamily\tiny,
stepnumber=1,
tabsize=4,
numbersep=5pt
}
\addbibresource{../../library/bibliography.bib}
\author{Fairphone Repair Group}
\title{Fairphone Repairability}
\date{\today}
\begin{document}
	\chapter{Business plan}
	\label{ch:business-plan}
	The following process has been provided to make a business plan \cite{BusinessCanvas}.
	\begin{enumerate}
		\item \textbf{Value Propositions}:
	
	The selected values are repairability, clarity and sustainability. Reparability is the fundamental most important one. Clarity is very important since the products are going to be used by many different types of customers and as this solution involves repairing products that get repaired by a specialist, it is essential for the designed products to have clarity when being used. Finally, sustainability is included because this is a value dear to Fairphone and therefore the designed products will meet their existing values.
	
	To help the customers, there is a repair service solution. This includes a network system from diagnosis to product repair. This is done through a redesigned phone-packaging (which serves as a repair mat) and a repair card which serves both as a repair tool and software/hardware diagnostics.
	
	The ultimate system is available to all the customers and offers a number of different ways to ease the diagnosing and repair process. The physical products share two coexisting functions and two individual functions. The redesigned box is a protection and organisation system for shipping, transport and repair of the phone, whereas the design card is used a connectible diagnosis device that gives a simple intractable diagnostics data (with capabilities to display more detailed information). They both share a role in the repair process offering a package to dissect and order the hardware of the phone.
	
	The system and products attend to the needs of multiple customers depending on the process that is needed and can be easily flexible to attend different scenarios, for instance:
	\begin{itemize}
		\item Direct TD (For Technophobes) 
		\item Repair card – TD – Box (speed)
		\item TD – Box (speed, for less tech-inclined users)
	\end{itemize}
	
		\item \textbf{Customer Segments}:
		\begin{itemize}
			\item For whom are we creating value?
			\item Who are our most important customers?
		\end{itemize}
		
	We are creating value for all customers who are experiencing problems with their Fairphone and that have potentially broken phones. The products aid the less technically capable by creating a way to easily diagnose and self-repair the phone and aid the more technically capable by increasing the total self-repair time. The system aids all customers by increasing the turn over time of the replacement parts and thus increases the speed of the repair.
	 
	The most important customers are our TD's. They are very much the bridge stones of our system creating a link between Fairphone and consumers without the use of middle men. They establish a way for the consumers to repair without long writing times and also provide a much more intimate ``help service''.
		
		\item \textbf{Customer Relationships}:
		\begin{itemize}
			\item What type of relationship does each of our Customer Segments expect us to establish and maintain with them?
			\item Which ones have we established?
			\item How are they integrated with the rest of our business model?
			\item How costly are they?
		\end{itemize}
		
		\item \textbf{Key Activities}:
		\begin{itemize}
			\item What Key Activities do our Value Propositions require?
			\item Our Distribution Channels?
			\item Customer Relationships?
			\item Revenue streams?
		\end{itemize}
		
		\item \textbf{Key Partners}:
		\begin{itemize}
			\item Who are our Key Partners?
			\item Who are our Key Suppliers?
			\item Which Key Resources are we acquiring from partners?
			\item Which Key Activities do partners perform?
		\end{itemize}
		
		\item \textbf{Key Resources}:
		\begin{itemize}
			\item What Key Resources do our Value Propositions require?
			\item Our Distribution Channels? Customer Relationships? Revenue Streams?
		\end{itemize}
		
		\item \textbf{Key Activities}:
		\begin{itemize}
			\item What Key Activities do our Value Propositions require?
			\item Our Distribution Channels?
			\item Customer Relationships?
			\item Revenue streams?
		\end{itemize}
		
		\item \textbf{Channels}:
		\begin{itemize}
			\item Through which Channels do our Customer Segments want to be reached?
			\item How are we reaching them now?
			\item How are our Channels integrated?
			\item Which ones work best?
			\item Which ones are most cost-efficient?
			\item How are we integrating them with customer routines?
		\end{itemize}
		
		\item \textbf{Cost Structure}:
		\begin{itemize}
			\item What are the most important costs inherent in our business model?
			\item Which Key Resources are most expensive?
			\item Which Key Activities are most expensive?
		\end{itemize}
		
		\item \textbf{Revenue Stream}:
		\begin{itemize}
			\item For what value are our customers really willing to pay?
			\item For what do they currently pay?
			\item How are they currently paying?
			\item How would they prefer to pay?
			\item How much does each Revenue Stream contribute to overall revenues?
		\end{itemize}
	\end{enumerate}
\end{document}