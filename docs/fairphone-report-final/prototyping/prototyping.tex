% !TeX spellcheck = en_GB
% !TEX program = xelatex+makeindex+bibtex
% !TeX TXS-program:pdflatex = txs:///xelatex %
\documentclass[final,a4paper]{report} %scrreprt of scrartcl
%!TEX program=xelatex+makeindex+bibtex
% Include all project wide packages here.
\usepackage{polyglossia}
\setmainlanguage{english}
\usepackage[strict,autostyle]{csquotes}
%\usepackage[svgnames]{xcolor}
\usepackage{graphicx}
%\usepackage{epstopdf}
%\usepackage{pdfpages}
\usepackage{caption}
\usepackage[list=true]{subcaption}
\usepackage{float}
\usepackage{standalone}
\usepackage{import}
\usepackage{tocloft}
%\usepackage{wrapfig}
%\usepackage{authblk}
%\usepackage{array}
%\usepackage{multicol}
%\usepackage{multirow}
\usepackage{booktabs}
\usepackage[toc,page,title,titletoc]{appendix}
\usepackage{xunicode}
\usepackage{fontspec}
%\usepackage{tikz}
%\usepackage{pgfplots}
\usepackage[binary-units]{siunitx}
\usepackage[parfill]{parskip}
\usepackage[pages=some]{background}
%\usepackage[absolute]{textpos}
\usepackage{titlepic}
%\pgfplotsset{compat=newest}
%\pgfplotsset{plot coordinates/math parser=false}
%\newlength\figureheight 
%\newlength\figurewidth
\usepackage{mathtools}
\usepackage{unicode-math}
\usepackage{rotating}
\usepackage{fancyhdr}
\usepackage[compact]{titlesec}
\usepackage{enumitem}
\usepackage{verbatim}
%\usepackage{blindtext}
%\usepackage[margin=3.5cm,headheight=35pt]{geometry}
\usepackage[
    backend=bibtexu,
	texencoding=utf8,
bibencoding=utf8,
    style=ieee,
    sortlocale=en_GB,
    language=auto
]{biblatex}
\usepackage{listings}
%\usepackage{wrapfig}
\usepackage{fullpage}
\newcommand{\includecode}[4][c]{\lstinputlisting[caption=#2, escapechar=, style=#1,label=#4]{#3}}
\newcommand{\superscript}[1]{\ensuremath{^{\textrm{#1}}}}
\newcommand{\subscript}[1]{\ensuremath{_{\textrm{#1}}}}


\newcommand{\chapternumber}{\thechapter}
\renewcommand{\appendixname}{Appendix}
\renewcommand{\appendixtocname}{Appendices}
\renewcommand{\appendixpagename}{Appendices}

\usepackage[hidelinks]{hyperref} %<--------ALWAYS LAST

\input{../../library/style.tex}
\addbibresource{../../library/bibliography.bib}
\author{Fairphone Repair Group}
\title{Fairphone Repairability}
\date{\today}
\begin{document}
\chapter{Prototyping}
	The main objective of the project is to present a product, with which the following steps can be undertaken: design, manufacturing, devising a business plan (with the use of a ``Business Model Canvas'' \cite{BusinessCanvas}), user testing and analysing the user feedback in order to further improve the product. It was finally decided to deliver two products, to be used together or separately at the users preference. The first prototype is also the most practical one. It's a new packaging for the new Fairphone which will have a specific purpose also after the delivery of the phone. It will aid the user in repairing the phone in case of hardware failures or wished updates. The second product is a business card which serves as a diagnosing tool and a screwdriver required for the disassembly of the phone. In this section, the respective previously described steps are discussed.
\end{document}