% !TeX spellcheck = en_GB
% !TEX program = xelatex+makeindex+bibtex
% !TeX TXS-program:pdflatex = txs:///xelatex %
\documentclass[final,a4paper]{report} %scrreprt of scrartcl
%!TEX program=xelatex+makeindex+bibtex
% Include all project wide packages here.
\usepackage{polyglossia}
\setmainlanguage{english}
\usepackage[strict,autostyle]{csquotes}
%\usepackage[svgnames]{xcolor}
\usepackage{graphicx}
%\usepackage{epstopdf}
%\usepackage{pdfpages}
\usepackage{caption}
\usepackage[list=true]{subcaption}
\usepackage{float}
\usepackage{standalone}
\usepackage{import}
\usepackage{tocloft}
%\usepackage{wrapfig}
%\usepackage{authblk}
%\usepackage{array}
%\usepackage{multicol}
%\usepackage{multirow}
\usepackage{booktabs}
\usepackage[toc,page,title,titletoc]{appendix}
\usepackage{xunicode}
\usepackage{fontspec}
%\usepackage{tikz}
%\usepackage{pgfplots}
\usepackage[binary-units]{siunitx}
\usepackage[parfill]{parskip}
\usepackage[pages=some]{background}
%\usepackage[absolute]{textpos}
\usepackage{titlepic}
%\pgfplotsset{compat=newest}
%\pgfplotsset{plot coordinates/math parser=false}
%\newlength\figureheight 
%\newlength\figurewidth
\usepackage{mathtools}
\usepackage{unicode-math}
\usepackage{rotating}
\usepackage{fancyhdr}
\usepackage[compact]{titlesec}
\usepackage{enumitem}
\usepackage{verbatim}
%\usepackage{blindtext}
%\usepackage[margin=3.5cm,headheight=35pt]{geometry}
\usepackage[
    backend=bibtexu,
	texencoding=utf8,
bibencoding=utf8,
    style=ieee,
    sortlocale=en_GB,
    language=auto
]{biblatex}
\usepackage{listings}
%\usepackage{wrapfig}
\usepackage{fullpage}
\newcommand{\includecode}[4][c]{\lstinputlisting[caption=#2, escapechar=, style=#1,label=#4]{#3}}
\newcommand{\superscript}[1]{\ensuremath{^{\textrm{#1}}}}
\newcommand{\subscript}[1]{\ensuremath{_{\textrm{#1}}}}


\newcommand{\chapternumber}{\thechapter}
\renewcommand{\appendixname}{Appendix}
\renewcommand{\appendixtocname}{Appendices}
\renewcommand{\appendixpagename}{Appendices}

\usepackage[hidelinks]{hyperref} %<--------ALWAYS LAST

\input{../../library/style.tex}
\addbibresource{../../library/bibliography.bib}
\author{Fairphone Repair Group}
\title{Fairphone Repairability}
\date{\today}
\begin{document}
\chapter{Introduction}
\label{ch:introduction}
In a world where technology is becoming a bigger part of our lives as the years go by, the mobile phone has risen to be on of the most used items in an average persons life. As a phone company like Fairphone, it is not only very important to make the customer happy with the their purchase, but also to keep them happy after they have bought it. The best way to do this is with good customer service. Fairphones' name states its purpose: everything in the company is done striving to meet fair trade requirements, creating an altruistic image for themselves and they are looking for ways to improve themselves in that aspect even further.

Fairphone has given the task to find a way to make the customer service a positive experience for the client. This is very important because in many cases the customer requires service after a negative experience with the product, for instance, breaking a component. This is the starting point of the journey.

First of all, a story board is developed, to present different scenarios. Right after the client has broken his phone, it should be possible for him to quickly and painlessly change the broken part in a reasonable amount of time. This is where the idea of ``Trusted Distributors'' come in. 

Replacing the broken part currently consists of bringing the phone to the store, and having it sent back to the manufacturer to replace the part. This is usually expensive, as if that isn't enough, it also takes long. With this investigation, methods of improving the efficiency of this process are looked at, in order to turn it into a more positive, economic and maybe even greener experience. However, the phone parts still have to be transported from the manufacturer to the customer but this topic will not be of interest in this report.

Improving all these aspects will provide the customer with a more positive experience and turning their frown into a smile by economically, quickly and efficiently repairing their phone.

Additionally, two products have been developed. One is small enough to fit in you walled and will aid the customer with the repairs, and the second one is a new packaging system that again aids the customer in dis/reassemble their phones.
\end{document}