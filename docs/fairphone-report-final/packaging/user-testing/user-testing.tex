% !TeX spellcheck = en_GB
% !TEX program = xelatex+makeindex+bibtex
% !TeX TXS-program:pdflatex = txs:///xelatex %
\documentclass[final,a4paper]{report} %scrreprt of scrartcl
%!TEX program=xelatex+makeindex+bibtex
% Include all project wide packages here.
\usepackage{polyglossia}
\setmainlanguage{english}
\usepackage[strict,autostyle]{csquotes}
%\usepackage[svgnames]{xcolor}
\usepackage{graphicx}
%\usepackage{epstopdf}
%\usepackage{pdfpages}
\usepackage{caption}
\usepackage[list=true]{subcaption}
\usepackage{float}
\usepackage{standalone}
\usepackage{import}
\usepackage{tocloft}
%\usepackage{wrapfig}
%\usepackage{authblk}
%\usepackage{array}
%\usepackage{multicol}
%\usepackage{multirow}
\usepackage{booktabs}
\usepackage[toc,page,title,titletoc]{appendix}
\usepackage{xunicode}
\usepackage{fontspec}
%\usepackage{tikz}
%\usepackage{pgfplots}
\usepackage[binary-units]{siunitx}
\usepackage[parfill]{parskip}
\usepackage[pages=some]{background}
%\usepackage[absolute]{textpos}
\usepackage{titlepic}
%\pgfplotsset{compat=newest}
%\pgfplotsset{plot coordinates/math parser=false}
%\newlength\figureheight 
%\newlength\figurewidth
\usepackage{mathtools}
\usepackage{unicode-math}
\usepackage{rotating}
\usepackage{fancyhdr}
\usepackage[compact]{titlesec}
\usepackage{enumitem}
\usepackage{verbatim}
%\usepackage{blindtext}
%\usepackage[margin=3.5cm,headheight=35pt]{geometry}
\usepackage[
    backend=bibtexu,
	texencoding=utf8,
bibencoding=utf8,
    style=ieee,
    sortlocale=en_GB,
    language=auto
]{biblatex}
\usepackage{listings}
%\usepackage{wrapfig}
\usepackage{fullpage}
\newcommand{\includecode}[4][c]{\lstinputlisting[caption=#2, escapechar=, style=#1,label=#4]{#3}}
\newcommand{\superscript}[1]{\ensuremath{^{\textrm{#1}}}}
\newcommand{\subscript}[1]{\ensuremath{_{\textrm{#1}}}}


\newcommand{\chapternumber}{\thechapter}
\renewcommand{\appendixname}{Appendix}
\renewcommand{\appendixtocname}{Appendices}
\renewcommand{\appendixpagename}{Appendices}

\usepackage[hidelinks]{hyperref} %<--------ALWAYS LAST

\input{../../../library/style.tex}
\addbibresource{../../../library/bibliography.bib}
\author{Fairphone Repair Group}
\title{Fairphone Repairability}
\date{\today}
\begin{document}
	\section{User testing}
	\label{sec:packaging-user-testing}
	User testing is an important part of designing a product. It serves as a stepping stone which provides certainty (positive or negative) of the current state of the product. It provides the designer(s) with valuable information about the design of the product, and very useful feedback to further improve it. In the case of the two previously described prototypes, they will be tested individually even though their functions are linked.
	
	The desire of the user testing to acquire a better understanding of the impulse use of the designed product. This is to gather information of user behaviour and refine the intuitive actions that the product causes. Test scenarios will be created in which the tester will have to use the currently designed product (V1.0 – VX) in a range of tasks that will test the visual, interface and structural design of the card. These results will be recorded and compared.
	
	Test 1 results
	
	Test 2 results
	
	Test 3 results
	
	Testing the box product on users
	
	Before testing the product on users, several points need to be clear.
	First of all it has to be clear to the users what they are supposed to do, therefore some information will have to be provided to the users regarding the product that they will have to test
	Then it needs to be clear how the results will be assessed. Therefore assessment criteria need to be defined. 
	
	\textit{The provided information regarding the product:}
	
	When handing over the product to the user, some additional information will be given. This will be due to the fact that it needs to be clear to the user what he is supposed to do. The information that will be provided to the user will be based on his knowledge in a real scenario.
	
	In a real scenario the user knows that his/her phone is broken and has used the diagnosing software to determine which exact component is malfunctioning. Right now the user has got the a new component which he is going to replace the broken component with.  
	Therefore the following will be told to the user that is about to perform the repair:
	\begin{itemize}
		\item he/she has a phone which is represented by three stacked layers of cardboard
		\item the screws are represented by push pins 
		\item the phone has a defect component which needs to be replaced.
		\item the defect component is marked with an X 
		\item the repair component is marked with a tick 
		\item in a real scenario the components would be fragile electronics, so be careful with them
		\item the box has to be used in the repair process
		\item you can only go to the next layer of the phone after you have disassembled all the parts from that layer\footnote{This is to simulate a real repairing process where you need to remove certain components before you can reach the next components.}
	\end{itemize}
	
	
	\textit{User testing assessment:} 
	
	We will assess the test on 2 criteria: Time and emotional state of user during the process.
	The time criteria will be assessed by determining how long it takes the user to perform the repair process. There will be 6 categories:
	\begin{itemize} 
		\item A = under 1 minutes 
		\item B = under 2 minutes
		\item C = under 3 minutes
		\item D = under 4 minutes
		\item E = under 5 minutes
		\item F = longer than 5 minutes
	\end{itemize} 
	In the case when the user falls under category F it will be assumed that something was wrong/unclear. The user will in that case be asked what hindered him in repairing the product.
	
	The ``emotional state of the user'' criteria will be assessed by asking the user how he/she felt during and after the repair process. If the user is experiencing any sort of positive emotions this will be a positive test result. In case the user is experiencing negative emotions he/she will be asked what caused these emotions. 
	
	These results will help us find the bottlenecks in the repairing process and help us remove them in order to make the repair process even easier, faster and more self explanatory. 
	
	The results from the user tests will be written down and put into a logbook. This logbook will show in which of the 3 categories of users this user belongs, namely young technically inclined, middle aged tech-savvy and elderly untechnical. It will show his/her age, name, time result  and emotional state and eventual causes for a long repair time or negative emotions. There will be a summary of how the repair process was performed and room for remarks from the user. There will also be room for side remarks
	
	Logbook:
	
	User 1
	Name:			Leo van den Hoed
	Age:			61
	Category:		Elderly untechnical
	Time:			A 
	Emotion:		positive
	Summary: 		The user did not use the box at all. He disassembled the phone and replaced the broken component very quickly.
	User remark:	After the repair after he was told that the box was meant to help him in the repair process he said he though that it was unnecessary. This is probably due to the fact that a cardboard phone does not look fragile and thus you tend to be less careful with than in the case you have real components.
	Side remark:	After this test a line is added to the information list that will be provided to the user, namely ''in a real scenario the components would be fragile electronics, so be careful with them''
	
	User 2
	Name:			Clara Arnolds	
	Age:			62
	Category:		Elderly untechnical
	Time:			C
	Emotion:		content, a bit surprised that the repair process is this easy
	Summary: 		The user performed the repair process in an organised manner, using the box in the exact way it is supposed to be used.
	User remark:	Not sure what to do with the broken component after removing it from the phone. 
		
	User 3 
	Name: 			Cobi van den Hoed	
	Age:			67
	Category:		Elderly untechnical
	Time:			B
	Emotion: 		calm 
	Summary:		The user performed the repair process quickly but once again without using the box.
	User remark:	After being told that the box was meant to aid in the repair process she stated that the box is not needed and only makes the process more time consuming. 
	Side remark:	We noticed that due to the fact that the prototype looks so simple, the users tend to skip using the box. So in order to get user test results on our prototype of the box the line was changed from''the box is meant to facilitate the repair process'' to ''the box has to be used during the repair process'' in the information that will be given to the users.

\end{document}